% ============= single column ==============
%\documentclass[a4paper,10pt]{article}
% ============= Double Column ==============
\documentclass[conference]{IEEEtran}
% ==========================================

\IEEEoverridecommandlockouts

% *** GRAPHICS RELATED PACKAGES ***
%
\ifCLASSINFOpdf
	
\fi

\usepackage[english]{babel}
\usepackage[utf8]{inputenc}

\usepackage{units}
\usepackage[tight,footnotesize]{subfigure}
\usepackage{graphicx}

\usepackage{tabularx}
\usepackage{booktabs}

\usepackage{comment}    %to use commnet command
\usepackage{setspace}

\usepackage{lipsum} % to use fig in multicolumn
\usepackage{acronym}

%Definition of the acronyms used in this paper
\acrodef{5G}{five generation}
\acrodef{CAPEX}{capital expenditures}
\acrodef{OPEX}{operating expenditures}

\renewcommand{\figurename}{Figure}

\newcommand{\PC}[1]{\ensuremath{\left(#1\right)}}
\newcolumntype{C}[1]{>{\centering\arraybackslash}m{#1}}
\newcolumntype{L}[1]{>{\arraybackslash}m{#1}}

% correct bad hyphenation here
\hyphenation{al-go-rithm allows op-tical net-works semi-conduc-tor ubi-quitous energy expe-riments another Fi-gures co-verage si-mulation decrea-sing speci-fically si-mulations dimensio-ning topo-logy mo-nitoring me-chanism acknow-ledgement diffe-rent pro-blem se-veral probability although communication access minimal expe-rience intero-perability effectively signalling simula-ting}

\begin{document}
    \bstctlcite{IEEEexample:BSTcontrol}

	% make the title area
	\title{CDN Slicing Resource Management in Edge Computing}

\author{
\IEEEauthorblockN{Eduardo~S.~Gama, Roger~Immich and Luiz~F.~Bittencourt}
\IEEEauthorblockA{Institute of Computing - State University of Campinas, Brazil \\ eduardogama@lrc.ic.unicamp.br, bit@ic.unicamp.br}}

	\maketitle

	\begin{abstract}
	    Integrated wireless-optical access networks based on \ac{RoF} has become one of the main technological trends in \ac{5G} networks due to its intrisic advantages such as high transmission capacity and immunity to electromagnetic interference. Moreover, it enables the processing of several radio frequency (RF) signals in a centralized entity, reducing both \ac{CAPEX} and \ac{OPEX}. 
	    %
	    This paper describes a testbed developed in the framework of the H2020 Futebol project and presents preliminary experimental results on the impact of using \ac{RoF} system for zigbee-based \ac{IoT} applications based on packet error rate and \ac{RSSI} values.
	    %
	    Furthermore, we highlight some challenges to the use of \ac{RoF} system in the \ac{IoT} landscape.
	    %RoF system to monitor the environment over a university campus (temperature, humidity, noise, etc.) using an optical infrastructure. 
	    %Among other aspects we will evaluate the performance of different multi-hop protocols and the efficiency of RoF and D-RoF (Digitized Radio over Fiber) technologies.
	\end{abstract}

    \begin{IEEEkeywords}
    	CDN, Content Delivery Networks, Cloud Computing, edge computing, fog computing, VNF.
    \end{IEEEkeywords}

    %This command resets all acronyms to not used. Useful after the abstract to redefine all acronyms in the introduction.
    \acresetall
    
    
\section{Introduction}
\label{sec:intro}

%Live streaming is becoming increasingly popular as more and more enterprises want to stream on the Internet to reach a world-wide audience. Common examples include radio and television broadcasts, live events with a global viewership, sporting events, and multimedia conferencing

%*******Introduction of the CDN technology and Fog networks**********
The Content Delivery Networks (CDNs) seeks to provide efficiently the content geographically closer to the end-user. The main objective of CDN cache copies of data (image, video, movie, etc) to speedup the computation. Nowadays, the video streaming services on the CDN providers represents the majority of the internet traffic, the cisco forecast \cite{Ma:WWW13} estimates that in 2021 70\% of all internet traffic will be dominated by the video-streaming, whereas for mobile devices this estimate respresent 78\% of all mobile data traffic. Corrently, traditional CDN providers are designed to delivery content through big datacenters, into the core networks, with static end-users devices and stable Internet link, which have large bandwith and low latency as main characteristics. These systems still don't lead in consideration the possibility to use the egde/fog network to improve their services. Trough the egde computing which has as main idea use small datacenter/devices closer to the end-user at the edge network to compute/storage end-user applications. This way, CDN systems may aprovetar new era of technology to melhorar seu serviço de entraga de conteudo sob demanda.
%Insight and perspectives for content delivery networks, 2016. https://www.cisco.com/c/en/us/solutions/collateral/service-provider/visual-networking-index-vni/mobile-white-paper-c11-520862.pdf

Due to the CDN providers have to maintain Quality of Experience (QoE) guaratees and low latency, besides the exchange of data between the users and network providers be growing up exponentially year-by-year, which could causing a time-varying shifts in content demand for in their videos, as well as increases the delay. Recent works [1] [2] [3] highlight fog/edge computing to handle with these aspects and attend a variety of mobile and static end-users devices efficiently. 
% [1] sensor, [2] JSAC, [3] On-the-fly QoE-Aware Transcoding in the Mobile Edge
This approach to multiple devices and datacenters (with different types of capacity), organized hierarchically capable to accomplish computation and storage operation,  closer to the user. This way, edge/fog computing has novel challenges for how data processing and users are managed to improve quality of service (QoS) and QoE, as well as reduce costs, as well [1]. 
Leading CDN providers such as Akamai, LimeLight, Level 3 and MaxCDN have made large investiments in edge servers in emerging markets like Asia, South America, and Africa.
Moreover, this paradigm has attracted big companies such as Intel, Qualcomm, IBM, Cisco which have mentioned a large growth in the network's edge in a near future.
Netflix, Google and Amazon with interest to improve yours CDN services in order to save bandwith and operational costs without violate QoE garantees. % Add referees about companies with interests in fog computing paradigm

According to [Slack 2002] "an adequate capacity and demand balance can generate high profits and satisfied customers, while the" wrong "balance can be potentially disastrous." Therefore, capacity planning of a system involves finding the balance between demand and available installed capacity. For this, it is necessary to know the behavior of the system, through the survey of the parameters and mathematical analysis of its future behavior and possible analysis of the reliability of the installed system as a measure of efficiency, in situations where demand is directly linked to the need for a service available.

%An CDN service consists of a set of services that can be instantiated virtually, through VNFs,  will be presented later in this paper.

%trancosder, streamer, cache and orchestrator where each of can be instantiate separatelly  is To deliver the content NFV   enables virtualization  of  network  functions  that  can  be  deployed  as virtual  machines  on  general  purpose  server  hardware  in  cloud environments,  effectively  reducing  deployment  and  operational costs.

%Os recentes surgiemntos de armazenamentos de provedores na nuvem tais como Amazon S3, Nirvanix e rackspace teêm aberto novas oportunidades de Prover CDNs custos efetivos.
%
%Recentes pesquisas seguem na direção de utilizar novas tecnologias emergentes 
%
%Devido ao deslocamento de parte do processamento para a borda da rede a decisão da computação executada e localização do armazenamento está sujeita a restrições da aplicação e localização geográfica do usuário. Enquanto o primeiro pode ser especificado de diferentes maneiras, como na forma de restrições de QoS, este último depende do comportamento humano (ou sistema autônomo). Em última análise, o comportamento do usuário determina o par de tempo/posição de um dispositivo que, juntamente com as restrições de QoS, pode ser usado para criar classes de aplicativos relevantes para o gerenciamento de recursos e agendamento em um ambiente de computação em névoa [11]. Ao reconhecer diferentes classes de aplicativos, este trabalho tem como objetivo principal estudar as diferentes politicas de escalonamento, algoritmos ou mecanismos para o balanceamento de carga no ambiente de computação em névoa, que completa tarefas reais dentro do prazo, aumenta a taxa de transferência e utilização da rede, mantendo a consistência dos dados com menos complexidade para atender a demanda atual de usuários finais

%Motivations of the paper
%Nowadays, there is a large amounts of work addressing CDN with datacenters, while much less work about cache hierarquies and the usage in fog computing on the edge network. The cache 

%A troca de dados entre usuário e cloud para videos em UHD satura e causa queda de redes backhaul,bem como aumenta o delay. Desta forma, é essencial implementar a computação na nuvem com a fog para trazer o trafego de dados, computação e funções da rede para a borda da rede, da forma que o usuário tenha a melor qualidade de experiancia possivel. 
%A arquitetura de computação em névoa é baseada em pequenos data centers distribuidos na borda da rede, nomeados de cloudlets, com o objetivo de evitar/reduzir o congestionamento do tráfego no núcleo da rede, reduzir a latência na comunicação da informação, além de facilitar a paralelização e/ou distribuição de procedimentos entre aplicações [3]. As cloudlets tem a capacidade de computação menor que os data centers usados no núcleo da nuvem e estão mais próximas aos usuários, mas eles podem contar com os data centers na nuvem sempre que necessário. Esta abordagem para vários data centers (com diferentes tipos de capacidade), hierarquicamente organizados, tem implicações sobre como processamento de dados e dos usuários são gerenciados para melhorar a qualidade do serviço (QoS - Quality of Service), bem como reduzir os custos, assim, trazendo novos desafios [][1].


%Objectives of the paper
% Address the content of this paper as questions to be answers
Whith the large interestes of the technological enterprises leaders in invests in edge servers.
This paper seek to discuss the most recent works in CDN networks, specially addressing emerged technologies, such as VNF and SDN which became feasible the CDN operations whithin the edge network, thereby provision CDNaaS in enviroments multi-tier environment into the edge. The architecture used to provision CDN services, and a better QoE to the user, and the caracteristics of each component that compound this architecture into to cloud. How it works in a cloud environment? How can content caching systems whithin the edge network could be made more efficient, scalable and cost-effective?
provide the technical elements to build mechanisms for cost- and quality-optimized CDN provision.


%Paper organization
This paper is organized as follows. 


%RoF systems consist of heterogeneous networks composed of wireless and optical links. Unlike traditional optical communications networks, in which a baseband signal is transmitted into the optical fibers, in RoF one or multiple analog carriers are transported into the fibers.
%The RoF link lies within the physical layer of the wireless system to be supported, and thus, it becomes an extension of the radio access domain. This enables the possibility to dynamically allocate radio resources, optimizing the spectrum utilization.
%
%Motivated by the advantages provided by the RoF technology and the ubiquity enabled by the Internet of Things (IoT), this experiment aims to develop an efficient RoF environmental monitoring system at the University of Campinas (UNICAMP) to monitor the environment in the university using an optical infrastructure already deployed with underused resources and dark fibers. Moreover, this experiment will deploy equipments at UFRGS to offer a configurable, distributed RoF testbed.


    
    \section{multi-tier Fog computing model}
\label{sec:fog-model}

 A uso da computação em névoa para a distribuição de conteudo CDN pode ser vital para a good QoS and QoE garantees. In these kinds of scenarios - that make up the fog network - may be any access points (AP) can be extended to also provide computing and storage services. The APs may be any device that offers wired/wireless network connection to the end-user, such as smart phones, tablets, laptops, connected vehicles.% Cite bit paper 

To clarify the possibilities, we first introduce a multi-tier fog computing model detailedly in Figure \ref{fig:fog-model}. As we can see, the higher tier is composed by the Provider that offers video streaming services (e.g. Netflix, Amazon, Youtube) located at the cloud, and makes either use of their own cache systems or leases one offered by CDN providers. %(... CDN providers: Akamai, CloudFlare, Rackspace, Amazon's AWS, MaxCDN CDN nodes)
These providers operate original content in datacenters at the WAN (long-haul netwoks). In fog network, the APs located at the fog are responsible to provisioning resources to CDN providers allocate their caches. Notice that the caches are to multi-hop away from the content provider/consumer. A very popular fashion to organize the cache placement content hierarchicaly, and bi-directional communication. In addition, each content pode ser splitted dynamically in a set of pieces in order to serve the end-users, taking into account different aspects, such as the mobility presented by the user and the possibility to predict the moving directing. 

With new possibilities being created to offer better services and the working of the internet. As long as the network becomes more robustness, the problem become more complex and new challenges arises to be solved. To tailored a CDN systems with multi-tiers environment into the fog, different characteristics have to be studied such as cache allocation, placement, replacement and selection caches, usually, making real-time decisions. As different cache sizes being allocation over the tiers to storage a range of content to provisioning a region. Also, an cache size in an AP device have to be able of deal a set of different content pieces, in order to the end-users may have QoE garaties. Thus, different level of granularity into the APs devices arises, if the granularity of certain areas becomes too fine, scalability issues begin to appear. Dividing the content of the fog in the network over the APs at the right level of granularity is a complex problem in its own right.
% map unit (Algorithmic Nuggets in Content Delivery)
 
% An network can be represented by a $(n+2)$-partite graph with $G = N_{c} \cup ( \bigcup_{i}^{N} V_{f}^{i}) \cup V_{d}$ 

\begin{figure}
	\centering	
    \includegraphics[width=\linewidth]{images/intro.pdf}
	\caption{Experimental setup for the spatial-based methodology.}
    \label{fig:fog-model}
\end{figure}

For retrieve a data afetwards a VM (or content, service, etc) being allocated or allready exists. The operations feitas pela aplicação podem ser receiver-driven, such as used in Content-Centric Network (CCN) and named data network (NDN) paradgm. 
Organizadas de hierarquiquamente, one content object name could be retrive trough an url name "ucla/videos/demo.png". For a large video that is shareded em a set o chunks, thereby  que uma requisiçao pode ser feita especificando um chunk do video  da seguinte foma "ucla/videos/demo.png/1".

Um provedor de conteudo possui a set of repositorios CDN geograficamente distribuidos. As requisições dos usuarios para conteudos geralmente são feitas atraves de HTTP GET requests. To provisioning the content, the CDN redirects the request through either DNS redirection [38] or IP-layer anycast [41]. 

O conteudo provido por repositorios CDN geograficamente distribuidos, visando estar proximo ao usuario. Procura melhorar o troughput da rede. Mas para isso o provedor do conteudo alvo.  CDN system


\begin{figure}
	\centering	
    \includegraphics[width=\linewidth]{images/router-processing-flow.pdf}
	\caption{State machine of rountig operation.}
    \label{fig:fog-model}
\end{figure}

    
\section{Business model perspective}

A flexibilidade ao utilizar da edge networks infrastructure oferecem new oportunidades para melhorar o servico, abordando metricas e perspectivas que geralmente são analisadas de forma independente. It can be seen in three different perspectives:

% Look at Offloading Content with Self-Organizing Mobile Fogs reference, present three respectives business ains and costs.

\begin{itemize}

\item \textbf{End-user perspective:} uninterrupted connectivity and communication services, smoth consumer experience.

\item \textbf{Content provider perspective:} connected intelligent 	transportation systems, road-side service units, sensors, and mission critical monitoring/traking services.

\item \textbf{Network operator perspective:}  scalable, energy- efficient, low-cost, uniformly-monitored, programmable, and secure communication infrastructure.

\end{itemize}

%% CDN providers: Akamai, CloudFlare, Rackspace, Amazon's AWS, MaxCDN CDN nodes


\section{Distribution of Information Centric}


\section{CDN Slicing Architecture}
\label{sec:cdn-slicing-archi}

This CDN service based on fog computing consider the scenarios presented in Fig. \ref{fig:fog-model}. This section is interested in make up an architecture capable to provisioning this service in on-demand manner. In order to investigated aspects as scalability over the live migration, granularity of different contents and providers, aside from mobile predictability considering the social medias and moving direction. Initially, the model at the Fig \ref{fig:intro} shows two layer components: the user device, fog nodes into the multi-tier APs and the cloud providers. 

The usage of CDN multi-tier fog platform rather than the cloud CDN have to be consider, mainly, the cost reduction.
The Edge server placement problem is responsavel de selecionar um nó na borda da rede. 
There will exists two kind of nodes in the network, which are both known and unknown nodes due to the flexibility presented by the edge. The edge node location could respresented by a tuple da $<< city, AS(Autonomous\ System)>>$. let us assume that in current scenario a set of files $S = \{s_{1}, s_{2}, ..., s_{n}\}$ 

The problem to cost minimization could be reducted to the problem "the optimal bandwidth allocation problem" (OBAP), described in [].

The cost of a solution could range of an agreement with a CSP. As the focus of CDN apps are high in/out of datas, this work leverage the concerns of price machines needed to the CDN utilization and the charge of fog  networks, aside from the cost of download and upload files in order to sync different nodes and distribute the files to end-users.

For provide a high-quality experience sharing into the same regional APs, the model 

For provide a high-quality experience general components have to take advantage of the moving direction presented by the user. This way, the monitoring and analytics modules may cooperate togheter with advanced machine learning, neural networks techniques for decision-making. Characteristics as users location, direction movement, users behaviour, content resquested, amount of stored data, data traffic needs, cloudlets capacity, newotrk capacity, and so on.

make location-based virtual machine migration feasible in the Fog computing paradigm.

to build pro a hierarchical, bi-directional computing infrastructure: edge devices communicate with cloudlets and cloudlets communicate with clouds. Cloudlets can also communicate with each other to perform data and process management in order to support application requirements, and to exchange Fog control/management data (such as user device and application state).

\begin{figure*}
	\centering%%
    \includegraphics[width=\linewidth]{images/arch-cloud-fog.pdf}
	\caption{CDN Slicing Architecture for the spatial-based methodology.}
    \label{fig:intro}
\end{figure*}

%an architecture thet offer CDN as a Service (CDNaaS) on demand over the end-users have to take some decisions, such as resource allocation, placement, real-time management decisions, maintaining the service level agreement.


al-time management decisions, maintaining the service level agreement.

With the possibility to provision algorithms as a set of function trough VNFs to provision end-user  some service, CDNaaS may be decouple in specific functions which will be orchestrate over the network to provision QoE garantees for end-users. Below presents an Architecture standard ETSI for CDN.

To deliver an efficient CDNaaS into an heterogeneos network environment with multi tier fog nodes, our envisioned architecture showed up in Fig 1 is composed by a cloud provider where is contented big datacenters, from here the when the user request a movie. The fog computing may be organized hierarquicaly by multiple levels from which serve como meio de prover o serviço de CDN. As can be seen in fig 1 an user 1 receiving a content from WiFi and the another user is receiving from both LTE and user 1.

The most recently CDN architectures highlight Network Function Virtualization (NFV) principle para permitir a execução de serviços especificos em servidores remotos. Accorndlying to [], NFV supports a wide range of services by orchestrating the VNF (virtual network function) deployment and operation across diverse computing, caching, and networking resources over the multi-tier fog nodes. Whereas SDN  paradigm decouples the control plane from the underlying data plane in a centralized manner. It is independent from NFV, but it can add more flexibility to the SFs. Therefore, a  hybrid  concept  may  exploit  both  NFV  and  SDN  at  the same  time.  Virtualization  concepts  are  integrated  in  different areas such as cloud and telecommunication for more services flexibility provisioning. This combination technology may be deployed as virtual machines capable to provision slices of CDN components on-the-fly in a dynamic manner [2][3]. This way, whenever CDN as a Service (CDNaaS) may be offered on demand for the end-users. In addition, the use of NFV and SDN jointly can take advantage of an platform prepared to deal with an multi-tier edge environment. %SDN  paradigm  then  decouples  the  control  plane  from  the underlying data plane in a centralized manner. It is independent from NFV, but it can add more flexibility to the SFs. Therefore, a  hybrid  concept  may  exploit  both  NFV  and  SDN  at  the same  time.  Virtualization  concepts  are  integrated  in  different areas such as cloud and telecommunication for more services flexibility provisioning.
%Logical partitions in NFV is also known as resource slicing, which divides the network into slices made of different resources and capacities so as to offer differen- tiated services for heterogeneous use cases and enable creation of customized services with fine control fea- tures of QoS [7, 8, 12, 13].

%\begin{figure}[!t]
%	\centering
%    \includegraphics[width=\linewidth]{images/archi-ETSI.pdf}
%	\caption{Experimental setup for the spatial-based methodology.}
%    \label{fig:intro}
%\end{figure}

which delivery the distributed content,

it is needed instantiate VNFs that may replace virtualization services among de nodes in defirent tiers of the network. This way 

With the Needed of download high quality multimedia content by the mobility users, this section introduces an architecture multi-tier composed by heterogeneous wireless network access (fig). Due to propogation of a In a network The devices  Due to the diference of range among the devices such as Cloud, ISP, LTE, WiFi, D2D. 

introduces an architecture composed of a cloud computing (Tier 1) together with multi-tier fog nodes (Tiers 2, 3, and 4), which work collaboratively to enable service migration for video distribution with QoE support. In such architecture, we consider fully connected and fully fog-enabled scenario, where fog nodes are hierarchically organized to provide video services for end-users. There may be widely distributed local fog nodes, e.g., mobile devices (i.e., Tier 4), where such fog node relays the video content via device-to-device (D2D) wireless communication for mobile devices with high and similar traffic demands could cooperate with each other to form a D2D network. The neighborhood fog node, e.g., Base Station (BS) or Access Point (AP) (i.e., Tier 3), supports a few dozen to perhaps a few hundred local fog nodes. Above these would be regional fog node, e.g., baseband unit (BBU) or Internet Service Provider (ISP) (i.e., Tier 2), managing city-wide coordination. On the top of such multi-tier architecture, there is the cloud (i.e., Tier 1). 


    \section{CDN Provisioning and Releated Work}
\label{sec:releated-work}

This section presents the main previous work on the employment of CDN solutions on the Cloud computing and other realted technologies. 

% CDN-as-a-Service Provision Over a Telecom Operator’s Cloud
% Managing QoS Constraints in a P2P-Cloud Video on Demand System.
% OpenCache: A Software-defined Content Caching Platform.


% [ICC'15] Joint Content-Resource Allocation in Software Defined Virtual CDNs
% [CLCN'17] Optimal and Cost Efficient Algorithm for Virtual CDN Orchestration
% [CLCN'16] Scalable and Cost Efficient Algorithms for Virtual CDN Migration
% [ComNet'17] OPAC: An optimal placement algorithm for virtual CDN
Hatem \textit{et al.} [1][2][3][4] highlight Software Defined Network (SDN) and Network Function Virtualization (NFV) principles into the cloud. The SDN/NFV-based approach allow to virtualization specific functions in remote servers. This way, the migrations of CDN services can be virtualized over different datacenters. Hatem et al. addresses orchestration and cache problem, its work develope an exact algorithm for deciding the optimal locations to place CDN functions. The proposed algorithm including content caching and request redirection is introduced with operatong system, network, and quality of experience constraints. Therefore, for managing the CDN is made by a centralized way, and use end-user as target to make Device-to-device communication but not explores mobility. End-users requests will redirected to an optimal edge cloud location, whithout a different multi-tier level edge devices. Llorca \textit{et al.} [ICC'15] propose a virtual cache network deployed fully in software over a programmable distributed cloud network infrastructure that can be elastically  consumed and optimized using global information about network conditions and service requirements called SDvCDN. This approuch address placement (facility location), routing (flow network) and resource allocation (network design) problems.

% [ICDCSW'16] Fog Cloud Caching at Network Edge via Local Hardware Awareness Spaces
% [Computer'15] A Cloud Visitation Platform to Facilitate Cloud Federation and Fog Computing
Zhanikeev \textit{et al.} [ICDCSW'16] proposes an caching technology distributed in 2-tier. Where the top layer is the original copy running in large-scale storage cloud, and the bottom tier are maintained by each participant of fog. The cloud has access to a number of fog nodes distributed regionally, and each regional network edge can balance the inter-cloud traffic load by keeping a portion of popular content at each local cloud. The network edge is splitted in two kinds of caches, the first one is an in-VM storage, implemented as files on virtual disk. The in-VM caching is volatile, either have to migrate with their caches or destroy them at each population upgrade. The second knd of cache is a storage facility outside of VM but inside a given fog. The pros of this cache are two-fold. First, it can be much larger than the in-VM cache. Secondly, the contents are persistent for the population in that fog cloud. The technology presented in previous work [Computer'15] is not limited to caching and storage in general, and can work for any generic service, including Hadoop environments, sensors, etc.

% [TNSM'17] CDN-As-a-Service Provision Over a Telecom Operator's Cloud
% [SIGCOMM'13] Pushing CDN-ISP collaboration to the limit
Frangoudis \textit{et al.} [1] and Frank \textit{et al.} [2] design an architecture for telco operator, which allows the interfaces and management tools to deploy a CDN infrastructure and lease it on demand. In [2] specifically design a prototype system called NetPaaS (Network Platform as a Service). In this case PaaS and IaaS services are provisioned by telco operator. The NetPaas support virtualization and physical CDN deployments, which allow the CDN operator can be in full control of the virtual resources. Whereas, in [1] offer a business model to the content provider lease a the CDN service in a Software-as-a-Service (SaaS) manner. Thus, the telco operator is capable to be on charge of the infrastructure and the CDN service. In Addition, the telco operator also may take better decision of the resource allocation. It should be noted that both works relies on the old and well-known technique called DNS forwarding [3] for load balancing – refer to [6] for a review of the various load balancing techniques.


% [JNCA'16] Hybrid multi-tenant cache management for virtualized ISP networks
% [CNSM'14] Proactive multi-tenant cache management for virtualized ISP networks.
Claeys \textit{et al.} [1] propose an Integer Linear Programming (ILP) formulation of multi-tenant content placement and server selection problem. The scenarioes tailored of the work was the Internet Service Providers (ISPs). The main objective is to maximize the hit ratio of cache content into the ISPs servers, thus minimizing the bandwith consuoption. to become more realistic the proposed model 
take into account the migrations overhead introduced by the frequent contents requested. The proactive and reactive placement strategies are studied. The proative approach mortou ter um desempenho melhor na migração de conteudo durante horas de pico e mais cache hits on the first request of popular content. However, para alcançar este desempenho é preciso ter uma forte precisão na predição de popularidade do conteudo, o que torna um complicado devido as caracteristicas do trafego VoD, que ocorre grande variação de trafego durante o tempo. To deal with these limitations, this paper proposes a hybrid cache management system. The espaços em cache são distribuidos gegraficamente para a execução da estrategia proatica, wherever occur unexpected pattern changes of request pattern the estrategia reativa é executada simultaneamente. 

% [INFOCOM'16] Dynamic Resource Orchestration for Multi-task Application in Heterogeneous Mobile Cloud Computing 
Qi Qi \textit{et al.} [I'16] propose a orhestrator framework to play offload workflows in heterogeneous Mobile Cloud Computing Environments. Which the workflow tasks are ddistributed at achieving maximal perfomance experienced by end users and minimal cost os cloud resources. The app responsible for realizar o offladof the tasks work separately, sem a possibilidade de reaproveitar as tarefas de workflows já executados. Este trabalho aborda aspectos importantes de predição de mobilidade and distribuir as tasks da melhora maneira possivel em diferentes redes utilizando o conseito de virtualização. The work [I'16] focuses an orquestração de workflows levando em consideração o custo de alocação, consumo de energia e a mobilidade do dispositivo. In CDN systems exitem  caracteristicas distintas entre offload de aplicações, such as cache hit ratio which is the compartilhar conteudo em comum entre usuarios finais, the timestamp de acordo com o popularity of content.

% [SENSORS'18] Service Migration from Cloud to Multi-tier Fog Nodes for Multimedia Dissemination with QoE Support

Rosario \textit{et al.} [1] apresenta uma arquitetura para servicos de migração ao vivo de VM da nuvem para multiniveis da fog. O cenario experimental a nuvem distribui o conteudo de video para os diferentes niveis da fog. A arquitetura é baseada no paradigma sdn para, distribuição de video com suporte a QoE. 
The work split the multi-tier fog in three tier in order to their cover, storage, upload and download capacity. Important aspects could be tailored to support generic content and IoT environments, besides work with both private and public clouds. A divisão da nuvem em multiniveis se dá pelas caracteristicas do  dispositivos conectado a nuvem, e não por qualquer interconexão entre esses aparelhos. The paper tem como focus prover tecnologias capazes de tornar este ambiente factivel, e melhorar o provisionamento de conteudo de servicos de stream de video.


\begin{table*}[ht]
\begin{center}
\begin{tabular}{|c|p{1.8cm}|p{1.5cm}|p{1.8cm}|c|p{2.3cm}|p{3.5cm}|}

\toprule
\multicolumn{7}{|c|}{Related Work}\\
\midrule
Paradigma & Referencia & Tipos de Conteudo & Mobilidade & tipo & Mecanismos & Problemas \\
\midrule
MEC & [8][15][16] & videos & Not & files & cache [8], cache+transcoder [15][16] & cache placement problem, content request load assignment \\
\midrule

Muti-cloud & [6][7][14] [17][19] & generic, video 	[17] & Not & files & cache[6][7][14] [19], transcoder & \\

\midrule

 & & & & & & \\

\bottomrule

\end{tabular}
\end{center}
\end{table*}

\begin{table*}[t]
\begin{center}
  \begin{tabular}{c|cc|cc|cc|cc}
\toprule
\multicolumn{9}{c}{Related Work}\\
\toprule
Attack & \multicolumn{2}{c|}{Apache} & \multicolumn{2}{c|}{Apache with seven} & \multicolumn{2}{c|}{nginx} & \multicolumn{2}{c}{nginx with seven}\\
\midrule
& Success Rate & TTS & Success Rate & TTS & Success Rate & TTS & Success Rate & TTS\\
\midrule
Slowloris & 0.0\% & $\infty$ & 98.7\% & 0.15s & 15.3\% & 0.00s & 96.5\% & 0.01s  
\\[2pt]
\midrule
HTTP POST & 0.0\% & $\infty$ & 97.3\% & 0.14s & -- & -- & -- & --   
\\[2pt]
\midrule
Slowread & 13.8\% & 1.99s &  97.2\% & 0.11s & 5.2\% & 1.29s & 96.7\% & 0.02s 
\\[2pt]
\midrule
Resurrected Slowloris & 31.9\% & 1.28s  & 95.6\% & 0.58s & 4.3\% & 0.00s  & 99.6\% & 0.01s 
\\[2pt]
\bottomrule
\end{tabular}

\end{center}
% \vspace{-3mm}
\caption{Experimental results with sevenslow, sevenmem, and the combination of sevenslow, sevenmem\ and sevencpu. The duration of all experiments was of at least 30 mins. We measured the Success Rate, Time-to-Service (TTS), Stable Memory and the Time to Stabilize (TT Stab.).}
\label{tb:resu-seven-siege}
% \vspace{-3mm}
\end{table*}


% [] Multitier Fog Computing With Large-Scale IoT Data Analytics for Smart Cities

% [ISCC'17] A Mobile Edge Computing-assisted Video Delivery Architecture for Wireless Heterogeneous Networks
Li \textit{et al.} [1] propose an Integer Linear Programming (ILP) formulation and heuristics for the problem of per user joint video quality and network selection in a multi- access heterogeneous network. 

% [book CDN - Chapter 11-12] Future directions about scalability
% Scalability means to deploy services where (in the cloud) they will be the most efficiently accessible. It may well be that deep deployments remain the most rational repository for services, but tools are required to analyze alternatives and compare their costs with their respective performance. At the same time, using multiple edge clouds or multitier clouds opens the path to new forms of elasticity, within and between clouds


% [MOBILECLOUD'14] A Virtualized, Programmable Content Delivery Network
Woo \textit{et al.} [MOBILECLOUD'14] propose an open platform for content
delivery, namely vCDN, that can support a wide range of delivery patterns. It is envisioned that (delivery) control applications would be written by service providers using the platform   for representing the delivery-related requirements on distributing their specific content w.r.t scale, responsiveness, security, and other properties. Specifically, for each control application, the platform can translate it to the form of an overlay network of cloud-based edge servers so as to satisfy the
application-specific requirements. In doing so, the centralized controller of software-defined networking (SDN) [3] is incorporated into the overlay network manage- ment, and thus building an overlay network and configuring its route, cache, and security functions can be transparently supported at runtime. We explore the combination of NDN and SDN in order to achieve programmability in the domain of content delivery

Em [6] proposes a function that allows CDN applications to discover local caching facilities dynamically, at runtime. For simplicity, analyzes the case when each location offers two caching option: VM-based for each app location-global shared by all local apps.

% [ICC'17] Content Delivery Network Slicing: QoE and Cost Awareness
Retal \textit{et al.} [ICC'17] propõe uma plataforma de \textit{CDN as a Service (CDNaaS)} onde os usuário podem criar um \textit{slice} de CDN incluindo cache, transcodificador e \textit{streamers}, em ordem de gerenciar uma quantidade de videos para seus usuários. (Aborda CDN na nuvem)
% [JSAC'18] Optimal VNFs placement in CDN Slicing over Multi-Cloud Environment
Benkacem \textit{et al.} [JSAC'18] introduce a CDNaaS platform whereby a user can create a CDN slice defined as a set of isolated distributed network of edge servers over multi-cloud domains where a edge server hosts a single VNF such as virtual cache, virtual transcoder, virtual streamer and a CDN-slice-specific coordinator for the life cycle management of the slice resources and also for managing uploaded videos and subscribers. This platform is designed to have the maximum level of flexibility for scaling out of down a CDN slice on top of different public and private Infrastructure as a Service (IaaS) such as Amazon AWS service, Microsoft Azure, Rackspace, and OpenStack-managed cloud. Furthermore, the platform employs mechanisms and algorithms that create cost-efficient QoE-aware CDN slices, involving an optimal placement taking into account the desired QoE level. Therefore, the ob- jective of this paper is to find an efficient cost of CDN slice respecting, on one hand, the CDN owner requirements in terms of QoE, and on the other hand, the cloud infrastructure and its cost.

%The multi-objective solutions give end users and application providers multiple choices and enable them to decide tradeoff policy according to situation. 

% [SmartIoT'17] Elastic Urban Video Surveillance System Using Edge Computing
[SmartIoT'17] propose Elastic resource allocation for video surveillance systems. The elasticity comes from an algorithm they propose to handle some emergency surveillance event (like tracking a criminal)  which requires  a  sudden  increase  of  computation  and  communication  resources  to  make  sure  that  all  the  possible  images are analyzed within a reasonable timeframe. When such an emergency event happens, network bandwidth allocation is reconfigured  and  computing  resources  are  reallocated  (by launching new VMs in the impacted zone and balancing the workload on nodes). When experimenting in their physical testbed, they verified that data propagation round-trip time is about 5 times lower with edge nodes close to the cameras compared  to  the  cloud.  They  also  found  that  the  time  for launching new VMs in the emergency mode is between one and two minutes, which they claim is acceptable in such a scenario.

%Review: Erase ref forward
% [Trans'17] Deadline  constrained  video  analysis  via  in-%%transit computational environments




The integrati Traditionally, \ac{RoF} system are mainly used to decrease the energy consumption of the infrastructure \cite{Hall2013}. However, for from the IoT perspective, this is not a key point in the utilization of the \ac{RoF} technology.

    
    
\section{Challenges to CDN Provisioning}
\label{sec:Challenges}

In this section, we highlight some challenges for the complete realization of the CDN for Edge/Fog concept.

\subsection{Impact of mobility on CDN Service}

Methodology for Measuring the Impact of RoF system on IoT applications performance

\subsection{Multi-tier into the edge/fog}

\subsection{Security}

\subsection{Energy Efficiency}

What are the technological challenges on energy consumption from both IoT nodes and infrastructure.
	
    
    \section{Conclusion}
\label{sec:Conclusion}


From the brief discussion of the provisioning CDN into the fog networks, and exciting skills such self-management and self-organization ainda precisam ser enfretados. Distribuir tais funcionalidades de alocação de conteudo em caches localizados multi-niveis da fog, tais contedos podem ser fatiados da melhor forma.
	
    
    \section{Acknowledgements}
\label{sec:Ack}

This work was supported by the European Union’s Horizon 2020 for research, technological development, and demonstration under grant agreement no. 688941 (FUTEBOL) and the Brazilian Ministry of Science, Technology and Innovation (MCTI) through RNP and CTIC.

	\bibliographystyle{IEEEtran}
	\bibliography{references}
	
\end{document}